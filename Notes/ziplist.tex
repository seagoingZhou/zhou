\documentclass[UTF8,a4paper]{ctexart}
\usepackage{amsmath}
\usepackage{indentfirst}
\usepackage{multirow}

\setlength{\parindent}{2em}
\usepackage{listings}
\usepackage{xcolor}
\begin{document}
	\title{Ziplist}
	\date{}
	\maketitle
	
	
	\section{Overall Layout}
		\begin{table}[!hbp]
		
		\caption{压缩列表的构成}
		\begin{tabular}{|c|c|c|c|c|c|c|c|}
			\hline
				zlbytes & zltail & zllen & entry1 & entry2 & ... & entryN & zlend \\
			\hline
				uint32\_t & uint32\_t & uint16\_t & & & & & uint8\_t \\
			\hline
		\end{tabular}

		\begin{center}
		\caption{压缩列表节点的构成}
		\begin{tabular}{|c|c|c|}
			\hline
			PrevEntryLength & encoding & content  \\
			\hline		
		\end{tabular}	
		\end{center}
	
		\end{table}
	
	\par 注意:ziplist存储于一段连续的内存空间,ziplist和ziplistEntry并未定义结构体
	
	\section{Details of code}
	
	ziplist源代码中定义了结构体zlentry:
	\definecolor{mygreen}{rgb}{0,0.6,0}
	\definecolor{mygray}{rgb}{0.5,0.5,0.5}
	\definecolor{mymauve}{rgb}{0.58,0,0.82}
	\lstset{ %
		backgroundcolor=\color{white},      % choose the background color
		basicstyle=\footnotesize\ttfamily,  % size of fonts used for the code
		columns=fullflexible,
		tabsize=4,
		breaklines=true,               % automatic line breaking only at whitespace
		captionpos=b,                  % sets the caption-position to bottom
		commentstyle=\color{mygreen},  % comment style
		escapeinside={\%*}{*)},        % if you want to add LaTeX within your code
		keywordstyle=\color{blue},     % keyword style
		stringstyle=\color{mymauve}\ttfamily,  % string literal style
		frame=single,
		rulesepcolor=\color{red!20!green!20!blue!20},
		% identifierstyle=\color{red},
		language=c++,
	}
	\begin{lstlisting}[language={[ANSI]C},numbers=left]
	typedef struct zlentry {
		unsigned int prevrawlensize; 
		unsigned int prevrawlen;    
		unsigned int lensize;        
		unsigned int len;            
		unsigned int headersize;     
		unsigned char encoding;
		unsigned char *p;            
	} zlentry;
	\end{lstlisting}
	这个结构体并不是entry节点的实际组织形式,而只是为了用于接收entry节点的信息,简化ziplist的操作
	
	\begin{itemize}
		\small\item[-] prevrawlensize: 编码前一个entry长度所需字节数
		\small\item[-] prevrawlen: 前一个entry长度
		\small\item[-] lensize: 编码entry长度所需字节数
		\small\item[-] headersize:头部长度
		\small\item[-] *p: 指向entry最前面的指针,即指向PrevEntryLength区域的指针
	\end{itemize}


	
\end{document}